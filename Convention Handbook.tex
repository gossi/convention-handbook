\documentclass[11pt]{report}

\usepackage[pdftex]{graphicx}   % to include graphics
\pdfcompresslevel=9 
\usepackage[pdftex,     % sets up hyperref to use pdftex driver
	plainpages=false,   % allows page i and 1 to exist in the same document
	breaklinks=true,    % link texts can be broken at the end of line
	colorlinks=true,
	pdftitle=Unicycle Convention Handbook
	pdfauthor=Thomas Gossmann
]{hyperref} 
\usepackage{thumbpdf}

% fullref
\newcommand{\fullref}[1]{\autoref{#1} \nameref{#1}}

% a bit styling
\usepackage{enumitem}
\setlist[itemize]{parsep=0pt}
\setlist[enumerate]{parsep=0pt}
\setlist[description]{parsep=0.5\baselineskip}
\setlength\parindent{0pt}

% metadata
\title{Unicycle Convention Handbook}
\author{Thomas Gossmann}
\date{}

\begin{document}
\maketitle

\setcounter{tocdepth}{1}
\tableofcontents

\chapter{Introduction}

This is a short guide on how to run your own (small) unicycle convention (up to 
(tilde) %todo tilde symbol
150 participants). This guide is split into four parts. The first contains 
information about the organization of such a convention and what you as a host 
must do. Second part is about the convention itself. How your program looks and 
what you need for it. Third part addresses the accounting for the convention and 
has some numbers ready for you to calculate with. The fourth part contains the 
specialities of a Freestyle Convention and what rules apply for such a 
convention.

\chapter{Organization}

This chapter focusses on the organizing things that happens mostly before the 
convention starts.

\section{Organizing Order}

Organizing a convention is not a linear process. Some tasks depend on others and 
some are interdependent. Although here is a rough order in which these steps may 
approached.

\begin{enumerate}
	\item There are two items you can parallelize at the beginning
		\begin{itemize}
			\item Feasibility study
			\item Get your staff together
		\end{itemize}
	\item Find locations
	\item Find suppliers for food & drinks
	\item Set up a website w/ registration
\end{enumerate}

All items on this list are described in detail in the following sections.

\section{Feasibility Study}

The first thing when starting to organize a convention is to run a feasibility 
study to check for yourself if you verify as a host. This includes:

\begin{itemize}
	\item checking the location - if there is any and if validates as a host for 
	your expected amount of audience.
	\item talking to friends ask them if they would support you and join the 
	organizing team.
	\item check your organizing structures whether you are legally eligible for 
	hosting the convention.
\end{itemize}

\section{Team Up!}

You better have a team around you, that handles various tasks to support you. 
Some important positions are:

\begin{description}
	\item [Purchase] \hfill \\
	A person or a team that handles all the purchases for materials that are needed 
	for such a convention.
	\item [Accommodation] \hfill \\
	A team that organizes all the on-site accommodation.
	\item [Account-Manager] \hfill \\
	A person keeping track of all the registration fees.
	\item [Online Presence] \hfill \\
	A person who is responsible for the online presence and takes care of the 
	online registration.
\end{description}
Remark: Multiple positions can be filled by the same person.

\subsection{Helper Plan}
You should create a plan which tasks need to be done. You can print this 
schedule and let helpers fill in their own name to claim those tasks. Your 
helpers need to have a task assigned, anyway they run around like beheaded 
chickens. Make sure 

\section{Locations}

Make sure your convention has good locations, here's what you'd better be 
looking for.

\subsection{Riding Location / Gym}

A riding location is a must. For freestyle events preferably a gym. For outdoor 
events good spots are required. Make sure the locations can host the expected 
amount of riders plus audience.

\subsection{Accommodation}
Athletes need to sleep somewhere. Most of them are comfortable with their 
sleeping mattress and sleeping bag, just provide them a stay for that. However, 
depending on your facilities you might provide them different options during 
registration. Here are some possibilities:

\begin{description}
	\item [Sleeping Hall/Gym] \hfill \\ 
		This would be the easiest type of accommodation from the perspective of a 
		participant. Offering sleeping places may also be included in the 
		registration fee.
	\item [External] \hfill \\
		Accommodation can forwarded to external suppliers. You can make a deal 
		with those on the expected over-night guests to get a discount for your 
		participants (e.g. Youth hostels or hotels). Depending on your external 
		suppliers and provided options by them, include them in the registration 
		with their respective additional costs.
\end{description}

\section{Food \& Drinks}
Athletes need to eat and drink.

\subsection{Food}
Food includes three meals per day: breakfast, lunch and dinner. What you offer 
may depend on your provided accommodation. If you house all participants for an 
over-night stay you should at least provide breakfast. Lunch and dinner can be 
outsourced to a party service of your choice. See \fullref{sec:Accounting}
with estimates for required calculations on those.

\subsection{Drinks}
If you decided to provide drinks, then make sure you have enough water 
(without sparkles). Provide cups and pencils, so participants claim one, write 
their name on it and use it throughout the day. Other drinks are bonus and are 
experienced as a present of those kind hosts.

\section{Insurance}

Make sure, you checked the insurance thing. It may be enough to hand this off to 
every participant. When you have rented gyms or other buildings, check that 
you comply with the fire emergency rules. 

\chapter{Convention Programm}

\section{Competitions}
For a small convention, choose your competitions wisely to not overload your 
event. Make your final choice on competitions and appoint a director for each of 
your competition who is responsible for managing it.

\section{Workshops}
Workshops are a good way to teach your participants new skills or get in contact 
with each other on shared activities. They are good as an addendum to your 
convention or even can turn up to be the whole concept.

\chapter{Accounting}
\label{sec:Accounting}

An account spreadsheet is provided to manage your financial situation along with 
managing your participants. Here is a step-by-step tutorial on how to use it.

\section{Understanding Costs}
There are two types of costs, you must understand first, in order to work with 
the provided spreadsheet. There is \textit{per person costs} and
\textit{fix costs}.

\begin{description}
	\item[Per Person Costs] \hfill \\
	Those are costs were you buy an item for a certain value and route this to the 
	participants.
	
	Example: You may rent a gym for a certain amount per person (i.e. 10€/per person).
	
	Color for a per person costs in the sheet is a light yellow.
	
	\item[Fix Costs] \hfill \\
	Those are costs that have a fixed purchase price.
	
	Example: You may rent a gym for a fixed price (i.e. 400€ for the weekend).

	Color for a per person costs in the sheet is a light orange.
	
\end{description}

\section{Defining Fees}

The first step would be to define fee classes for your participants on the 'Fee' 
sheet. The first class 'Staff' is already available and should be kept. Assign 
this class for all your helpers, they don't need to pay for anything.

\subsection{Modularize your Fees with Packages}

It will be easier to define packages first and then combine them into classes. 
On the 'Fee' sheet start with defining packages on the right table first. 
Packages may be something like 'Breakfast Sa' for breakfast on saturday and a price
in the next column. Another package may be 'Sleeping Fr to Sa'. Now these two 
packages are of different cost types. The spreadsheet isn't clever enough to 
assume what type of cost these packages are, so you must manually assign a 
color. Even then, this is just for you, so you can visually inspect what type 
those packages are.

Example: The breakfast package is a fixed cost and you should give it an orange 
background, because you don't route the costs for all the breakfast ingredients 
to the participant. On the other hand, the sleeping option is a per person 
costs, because you may rent the sleeping gym for 10€ per person.

\section{Building up your Classes}

Now that you defined your packages it is time to build classes from them. Give 
the package a descriptive name and then you must calculate two prices from it.

\begin{description}
	\item [Price] \hfill \\
	This is the final price a participant will pay. Include the the base price on 
	top of the two tables plus the packages you defined earlier.
	
	% TODO: squarebraket spacing look ugly
	\item [Price [ per person costs only] ] \hfill \\
	This almost the same price as above but including the per person 
	costs only (You have correctly colored your packages above, do you?) - we'll
	need these two different values later.
\end{description}

\section{Planning}

Planning happens before the convention to find the correct price for your 
classes. At the end of the planning phase, you will have your base price 
settled.

Before going into the planning tab you must define your expected expenditures. 
On the 'Expenditure' sheet there are two tables. One for fix costs the other for 
per person costs. On the latter, you end up listing all the items you defined as 
per person cost packages with their respective price as value. However, this time 
you will need it for the final accounting, so you can leave out the amount (as 
you don't know the final number yet) and ignore the total price for now.

More important are the fix costs you are about to expect. This may include 
salary for trainers, drinks, breakfast, audio equipment, dinner, lunch, etc. and 
don't forget the miscellaneous item as this is were all the random things end up 
you forget in the early planning. Now there are two columns following each item. 
The first is the value you plan to spend for this item and the second column is 
used later for the actual accounting. 

Preparation for planning is done. On the 'Planning' sheet you will see the 
classes and the expenditure automatically appear with their assigned values. On 
the lower right table you may enter additional income, such as sponsoring. Next 
up is probably the most important step of the planning phase. It's time to find 
the price for each fee class. First is, you enter the amount of expected 
participants in the lower right table on the 'Planning' sheet. 
This step is a bit tricky as 
you might not know how much people will attend for sure. You must define a 
number of attendees at which you will make profit and if fewer attend you end up 
paying for your own convention. It is better to calculate with less people 
attending than you are actually hoping to come.
Second is to modify the 'Base Price' value on the 'Fee' sheet. Raise or lower 
the value to fit your needs, your fee classes are settled now - hooray!

\section{Managing your Participants}

On the 'Participants' page you can add all the participants of your convention. 
Age and Sex are two formal columns, feel free to fill them out. On the 
registration columns, select the class for the participant and the appropriate 
fee will appear. Fill in the paid column when you receive a payment and the 
balance column indicates if the participants has fullfilled his payment duties.

\section{Final Accounting}

TODO: What needs to be done after the convention.

\chapter{Freestyle Convention}

The concept for a freestyle convention was invented by Felix Dietze and Thomas 
Gossmann. Mainly because at that time there were only freestyle competitions.
At competitions only one or two riders are performing while the others aren't 
allowed to. We wanted the opposite, we wanted a gym full of freestylers riding, 
laughing, learning, talking and sharing unicycling with each other. That's why 
we invented the concept of a freestyle convention.

The more a participant needs to organize for a convention, aside from registration, 
the less likely people are willed to join. From other conventions we identified two 
things we wanted to neglect for the freestyle convention.
First is the need to manage your over-night stay on your own. In our opinion the 
host should provide accommodation. As freestylers we often declare gyms as our 
home and providing a place in a gym for your mattress and sleeping bag is 
sufficient enough. Also people stay in one place and socialize in the evening.
Second is food: When people need to manage food on their own, they form 
groups to organize something but a good friend sometimes is left out just because 
he wasn't at the right place at the right time. In our opinion the food is 
provided by the host, keeping people together let them eat all at the same time. 
In total this will lead to more time participants can spent with each other.

\section{Freestyle Convention Guidelines}

That leads us to the creation of guidelines for a freestyle convention:

\begin{itemize}
	\item Host provides a place to sleep (a gym is absolutely fine)
	\item Host provides food, this includes breakfast, lunch and dinner
	\item Host provides drinks (water is absolutely fine; more is bonus)
	\item A weekend full of workshops, because riders are coming to learn from 
	others
	\item No official competitions (though, small, spontaneous and funny ones in 
	form of workshops are allowed)
\end{itemize}

\end{document}
