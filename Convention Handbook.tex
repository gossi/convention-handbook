\documentclass[11pt]{report}

\usepackage{ifpdf}
\ifpdf 
    \usepackage[pdftex]{graphicx}   % to include graphics
    \pdfcompresslevel=9 
    \usepackage[pdftex,     % sets up hyperref to use pdftex driver
            plainpages=false,   % allows page i and 1 to exist in the same document
            breaklinks=true,    % link texts can be broken at the end of line
            colorlinks=true,
            pdftitle=Convention Handbook
            pdfauthor=Thomas Gossmann
           ]{hyperref} 
    \usepackage{thumbpdf}
\else 
    \usepackage{graphicx}       % to include graphics
    \usepackage{hyperref}       % to simplify the use of \href
\fi 

\title{Convention Handbook}
\author{Thomas Gossmann}
\date{}

\begin{document}
\maketitle
\tableofcontents

\chapter{Introduction}

This is a short guide on how to run your own (small) unicycle convention (up to 
~150 participants). This guide is split into three parts. The first contains 
information about the organization of such a convention and what you as a host 
must do. Second part is about the convention itself. What program is running and 
what you need for it. Third part addresses the accounting for the convention and 
has some numbers ready for you to calculate with. The fourth part contains the 
specialities of a Freestyle Convention and what rules apply for such a 
convention.

\chapter{Organization}

\section{Organizing Order}

Let's roughly go through the steps for organizing a small unicycle convention.

\begin{enumerate}
	\item There are two items you can parallelize at the beginning
		\begin{itemize}
			\item Feasibility study
			\item Get your stuff together
		\end{itemize}
	\item Find locations
	\item Find suppliers for food & drinks
	\item Set up a website w/ registration
\end{enumerate}

All items on this list are described in detail in the following sections.

\section{Feasibility Study}

The first thing when starting to organize a convention is to run a feasibility 
study to check for yourself if you verify as a host. This includes:

\begin{itemize}
	\item checking the location
	\item talking to friends ask them if they would support you and join the team
	\item check your organizing structures whether you are legally eligible
\end{itemize}

\section{Team Up!}

You better have a team around you, that handles various tasks to support you. 
Some important positions are:

\begin{description}
	\item [Purchase] \hfill \\
	A person or a team that handles all the purchases for materials that are needed 
	for such a convention.
	\item [Accomodation] \hfill \\
	A team that organizes all the on-site accomodation.
	\item [Account-Manager] \hfill \\
	A person keeping track of all the registration fees.
\end{description}

Remark: Multiple positions can be filled by the same person.

\section{Locations}

Make sure your convention has good locations, here's what you'd better be 
looking for.

\subsection{Riding location / Gym}

A riding location is a must. For freestyle events preferrably a gym. For outdoor 
events good spots are required.

\subsection{Accomodation}
Athletes need to sleep somewhere. Most of them are comfortable with the sleeping 
mattress and their sleeping bag, just provide them a stay for that. However, 
depending on your facilities you might provide them different options during 
registration. Here are some possibilities:

\begin{description}
	\item [Sleeping Hall/Gym] \hfill \\ 
		This would be the easiest type of accomodation from the perspective of a 
		participant. Offering sleeping places may also be included in the 
		registration fee.
	\item [External] \hfill \\
		Accomodation can forwarded to external suppliers. You can make a deal with 
		those on the expected over-night guests to get a discount for your 
		participants (e.g. Youth hostels or hotels). Depending on your external 
		suppliers and provided options by them, include them in the registration 
		with their respective additional costs.
\end{description}

\section{Suppliers}

\section{Food \& Drinks}
Athletes need to eat and drink.


\chapter{Convention Programm}

\section{Workshops}
Workshops are a good way to teach your participants new skills or get in contact 
with each other on shared activities. They are good as an addendum to your 
convention or even can turn up to be the whole concept.

\chapter{Accounting}

\chapter{Freestyle Convention}

The concept for a freestyle convention was invented by Felix Dietze and Thomas 
Gossmann. Mainly because at that time there were only freestyler competitions.
At competitions only one or two riders are performing while the others aren't 
allowed to. We wanted the opposite, we wanted a gym full of freestylers riding, 
laughing, learning, talking and sharing unicycling with each other. That's why 
we invented the concept for a freestyle convention.

The more you need to organize as a participant the less likely people are 
joining. From other conventions we identified two things we wanted to neglect 
for the freestyle convention.
First is the need to manage your over-night stay on your own. In our opinion the 
host should provide accommodation. As freestylers we often declare gyms as our 
home and providing a place in a gym for your mattress and sleeping bag is 
sufficient enough. Also people stay in one place and socialize in the evening.
Second is food: When people need to manage their food on their own, they form 
groups to organize something but a good friend sometimes is left out just because 
he wasn't at the right place at the right time. In our opinion the food is 
provided by the host, keeping people together let them eat all at the same time 
which will lead to more time they can spent together with many others on their 
unicycles.

\section{Freestyle Convention Guidelines}

That leads us to the creation of guidelines for a freestyle convention:

\begin{itemize}
	\item Host provides a sleeping place (a gym is absolutely fine)
	\item Host provides food, this includes breakfast, lunch and dinner
	\item Host provides drinks (water is absolutely fine; more is bonus)
	\item A weekend full of workshops, because riders are coming to learn from 
	others
	\item No official competitions (though, small and funny ones in form of workshops are allowed)
\end{itemize}

\end{document}
